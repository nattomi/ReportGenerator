\documentclass{scrartcl}

\usepackage{fullpage}
\usepackage[utf8]{inputenc}
\usepackage{amsmath}
\usepackage{amssymb}
\usepackage{amsthm}
\usepackage{hyperref}
\usepackage{float}
\usepackage{verbatim}
\usepackage{color}

\usepackage{tikz}
\usetikzlibrary{positioning, shapes, arrows}

\newtheorem{mydef}{Definition}
\newtheorem{remark}{Remark}

\title{Evaluating the performance of users}
\subtitle{of the otu.lea system}
\author{Tamás Nagy\\ \small nattomi@gmail.com}

\begin{document}
\maketitle
\section{Introduction}\label{sec:intro}
The performance of a user during a test-session (regardless whether the test was interrupted or not) can be summarized in a so-called \emph{mark table}. The formal definition of a mark table will be given later below, for now we just give an example for clarity:

\vspace{.5cm}
\begin{tabular}{lllllll}
timestamp	& subject	& level	& task	& subtask	& alphaid	& mark\\
2014\_11\_8\_10\_6\_59	& Schreiben	& Einfach	& 2.2.02\_III	& 2.2.02\_III\_10b	& 2.1.13	& 1\\
2014\_11\_8\_10\_6\_59	& Schreiben	& Einfach	& 2.2.02\_III	& 2.2.02\_III\_10c	& 2.1.08	& 1\\
2014\_11\_8\_10\_6\_59	& Schreiben	& Einfach	& 2.2.02\_III	& 2.2.02\_III\_11	& 2.2.08	& 0\\
2014\_11\_8\_10\_6\_59	& Schreiben	& Einfach	& 2.2.02\_III	& 2.2.02\_III\_12	& 2.2.08	& 1\\
2014\_11\_8\_10\_6\_59	& Schreiben	& Einfach	& 2.2.02\_III	& 2.2.02\_III\_13	& 2.1.07	& 0\\
2014\_11\_8\_9\_58\_0	& Schreiben	& Einfach	& 2.1.01	& 2.1.01\_1a	& 2.1.05	& 1\\
2014\_11\_8\_9\_58\_0	& Schreiben	& Einfach	& 2.1.01	& 2.1.01\_1b	& 2.1.14	& 1\\
2014\_11\_8\_9\_58\_0	& Schreiben	& Einfach	& 2.1.01	& 2.1.01\_2a	& 2.1.05	& 0\\
2014\_11\_8\_9\_58\_0	& Schreiben	& Einfach	& 2.1.01	& 2.1.01\_2b	& 2.1.14	& 0\\
\end{tabular}

\section{Definitions of some sets of concern}

\begin{mydef}
Throughout this document the sets of symbols
\begin{equation}
\Delta^+=\{1,2,3,4,5,6,7,8,9\}
\end{equation}
and
\begin{equation}
\Delta=\{0\}\cup\Delta^+
\end{equation}
are going to be referred to as \emph{positive digits} and \emph{digits}, respectively. The set of symbols
\begin{equation}
\begin{split}
\Omega=\Delta \cup \{.,\_\}\cup\{a,b,c,d,e,f,g,h,i,j,k,l,m,n,o,p,q,r,s,t,u,v,w,x,y,z,ä,ö,ü,ß\} & \cup\\
\{A,B,C,D,E,F,G,H,I,J,K,L,M,N,O,P,Q,R,S,T,U,V,W,X,Y,Z,Ä,Ö,Ü\} & \\
\end{split}
\end{equation}
is going to be referred to as \emph{alphabet}.
\end{mydef}

\begin{remark}
It is trivial from the definition that $\Delta^+\subset\Delta\subset\Omega$.
\end{remark}

\begin{mydef}
Elements of the set $$\mathcal{W}_n=\underbrace{\Omega\times \Omega\times\ldots \times \Omega}_{n}$$ are called \emph{words of length n over the alphabet $\Omega$}. We define the \emph{set of words over the alphabet $\Omega$} as $$\mathcal{W}=\bigcup_{n=1}^{\infty}\mathcal{W}_n.$$
\end{mydef}

\begin{mydef}
Elements of the set
$$\mathcal{I}=\Delta^+\cup\bigcup_{n=1}^{\infty}(\Delta^+\times\underbrace{\Delta\times \Delta\times\ldots \times \Delta}_{n})$$
are called \emph{integer words}.
\end{mydef}

Some examples of integer words are $(1,0)$, $(1,0,2)$, but f.i. $(0,0,1,2)$ is not a valid integer word.

\begin{mydef}
Let $A,B\subset\mathcal{W}$ be arbitrary sets. We call the set 
$$A.B=A\times\{.\}\times B\subset\mathcal{W}$$
the \emph{dotproduct} of those sets.
\end{mydef}

In the upcoming subsections we define some special subsets of $\mathcal{W}$.


\subsection{$\mathcal{T}$ - the set of timestamps}
I will give an exact definition later, for now it is enough to know that $\mathcal{T}\subset\mathcal{W}$.
\subsection{$\mathcal{S}$ - the set of subjects}
\begin{mydef}
The \emph{set of subjects} is defined as
\begin{equation}
\mathcal{S}=\{(L,e,s,e,n),(S,c,h,r,e,i,b,e,n),(S,c,h,p,r,a,c,h,e),(M,a,t,h,e)\}.
\end{equation}
\end{mydef}

\subsection{$\mathcal{L}$ - the set of levels}
\begin{mydef}
The \emph{set of levels} is defined as
\begin{equation}
\mathcal{L}=\{(E,i,n,f,a,c,h),(M,i,t,t,e,l),(S,c,h,w,e,r)\}.
\end{equation}
\end{mydef}

\subsection{$\mathcal{K}$ - the set of tasks}
I need more details in order to give an exact definition, for now it is enough to know that $\mathcal{K}\subset\mathcal{W}$.

\subsection{$\mathcal{U}$ - the set of subtasks}
I need more details in order to give an exact definition, for now it is enough to know that $\mathcal{U}\subset\mathcal{W}$.

\subsection{$\mathcal{A}$ - the set of alpha ids}
\begin{mydef}
Consider the following recursion of sets in $\mathcal{W}:$
\begin{equation}
\begin{split}
\mathcal{A}_0= & \mathcal{I}\\
\vdots &\\
\mathcal{A}_{n} = &\mathcal{A}_{n-1}.\mathcal{I} \mbox{ for } n>1.\\
\end{split}
\end{equation}
The set $$\mathcal{A}=\bigcup_{n=0}^{\infty}\mathcal{A}_n\subset\mathcal{W}$$ is called the \emph{set of alpha ids}. 
\end{mydef}

\begin{remark}
This definition doesn't describe the reality perfectly because in otu.lea some there are some alpha ids which are outside of $\mathcal{A}$, such as 2.1.08 (or $(2,.,1,.,0,8)$) in the example of section \ref{sec:intro}. However, it is rather an inconsistency in the implementation than in the definition above.
\end{remark}

Some examples of alpha ids are $(1,.,2,.,10)$ or $(10,.,2)$, but f.i. $(10,.,0,.,9,.,10)$ or $(a,.,10,.,b)$ are not valid alpha ids. In the remaining part of the subsection we introduce an ordering on the set $\mathcal{A}$. $\prec_{\mathcal{A}}$ or $\prec^{\mathcal{A}}$.


\section{Mark tables}
Let $A$ denote the letters of the German alphabet (including numbers and special characters). Let us denote by $\mathcal{S}(A)$ the words that can be created from $A$, i.e.
$$\mathcal{S}(A)=\cup_{k=1}^{\infty}A^k,$$
where $A^n$ stands for $\underbrace{A\times A\times\ldots \times A}_{n}$. 

At first we define a further subset of $\mathcal{S}(A)$, namely the subset of alphaids.

Let $B=\{0,1\}$ and $$\mathcal{M}=\underbrace{\mathcal{S}(A)}_{\mbox{timestamp}}\times\underbrace{\mathcal{S}(A)}_{\mbox{subject}}\times\underbrace{\mathcal{S}(A)}_{\mbox{level}}\times\underbrace{\mathcal{S}(A)}_{\mbox{task}}\times\underbrace{\mathcal{S}(A)}_{\mbox{subtask}}\times\underbrace{\mathcal{S}(A)}_{\mbox{alphaid}}\times \underbrace{B}_{\mbox{mark}}.$$ A \emph{mark table} is an element in $\mathcal{S}(\mathcal{M})$, but we can also think about it as a finite sequence $M=(m_i)_{i\in I}$ in $\mathcal{M}$ with $I=\{1,2,\ldots,n\}$ for some $n<\infty$. This definition could be even more fine-tuned but it's going to be sufficient for the use cases arise in this document.

\section{Evaluation modes}
We present two different evaluation methods here, one is called ``Das kann ich!'' (\emph{mode A1}) and the other is called ``Das kann ich bald wenn ich moch ein wenig übe'' (\emph{mode A2}). The evaluation methods are basically mappings from $\mathcal{S}({\mathcal{M}})$ to $\mathcal{P}({\mathcal{S}(A)})$, where $\mathcal{P}(\cdot)$ denotes the power set of a set.

\section{Description of evaluation mode A1}
We start from a mark table $M=(m_i)_{i\in I}$. Let's define the relation $\sim$ over $I$ as follows:
\begin{equation}
i\sim j \Leftrightarrow m_i^{(6)}=m_j^{(6)}.
\label{eq:part}
\end{equation}
It is easy to see that (\ref{eq:part}) is an equivalence relation and as such it defines a partition over $I$, i.e. there exist some nonempty, disjoint sets $I_1,I_2,\ldots,I_r$ such that $$I=\cup_{i=1}^r I_i$$ and $i$ and $j$ are in $I_k$ if and only if $i\sim j$ i.e. $m_i^{(6)}=m_j^{(6)}$. Moreover, this resolution of $I$ is unique. Let us choose an element $i_k\in I_k$ (i.e. a \emph{representative}) for each $k=1,2,\ldots,r$.

Let $t\in [0,100]$ denote our predefined threshold for fullfilling a competency. Then, our preliminary result set is defined by 
$$R=\{m_{i_k}^{(6)}: \frac{\sum_{i \in I_k}m_{i_k}^{(7)}}{|I_k|}\geq 100^{-1}t, k=1,2,\ldots r\}.$$
Let 
\end{document}
