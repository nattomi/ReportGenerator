\documentclass{scrartcl}

\usepackage{fullpage}
\usepackage[utf8]{inputenc}
\usepackage{amsmath}
\usepackage{amssymb}
\usepackage{amsthm}
\usepackage{hyperref}
\usepackage{float}
\usepackage{verbatim}
\usepackage{color}

\usepackage{tikz}
\usetikzlibrary{positioning, shapes, arrows}

\newtheorem{mydef}{Definition}
\newtheorem{remark}{Remark}

\title{Evaluating the performance of users}
\subtitle{of the otu.lea system}
\author{Tamás Nagy\\ \small nattomi@gmail.com}

\begin{document}
\maketitle
\section{Introduction}\label{sec:intro}
The performance of a user during a test-session (regardless whether the test was interrupted or not) can be summarized in a so-called \emph{mark table}. The formal definition of a mark table will be given later below, for now we just give an example for clarity:

\vspace{.5cm}
\begin{tabular}{lllllll}
timestamp	& subject	& level	& task	& subtask	& alphaid	& mark\\
2014\_11\_8\_10\_6\_59	& Schreiben	& Einfach	& 2.2.02\_III	& 2.2.02\_III\_10b	& 2.1.13	& 1\\
2014\_11\_8\_10\_6\_59	& Schreiben	& Einfach	& 2.2.02\_III	& 2.2.02\_III\_10c	& 2.1.08	& 1\\
2014\_11\_8\_10\_6\_59	& Schreiben	& Einfach	& 2.2.02\_III	& 2.2.02\_III\_11	& 2.2.08	& 0\\
2014\_11\_8\_10\_6\_59	& Schreiben	& Einfach	& 2.2.02\_III	& 2.2.02\_III\_12	& 2.2.08	& 1\\
2014\_11\_8\_10\_6\_59	& Schreiben	& Einfach	& 2.2.02\_III	& 2.2.02\_III\_13	& 2.1.07	& 0\\
2014\_11\_8\_9\_58\_0	& Schreiben	& Einfach	& 2.1.01	& 2.1.01\_1a	& 2.1.05	& 1\\
2014\_11\_8\_9\_58\_0	& Schreiben	& Einfach	& 2.1.01	& 2.1.01\_1b	& 2.1.14	& 1\\
2014\_11\_8\_9\_58\_0	& Schreiben	& Einfach	& 2.1.01	& 2.1.01\_2a	& 2.1.05	& 0\\
2014\_11\_8\_9\_58\_0	& Schreiben	& Einfach	& 2.1.01	& 2.1.01\_2b	& 2.1.14	& 0\\
\end{tabular}

\section{Definitions of some sets of concern}

Throughout this document the sets of some symbols are going to be referred to as a \emph{alphabet}. Some examples of a alphabet includes
$$\Delta^+=\{1,2,3,4,5,6,7,8,9\} \mbox{ and }\Delta=\{0\}\cup\Delta^+$$
which are going to be referred to as \emph{positive digits} and \emph{digits}, respectively. Another example is the set 
\begin{equation}
\begin{split}
\Omega=\Delta \cup \{.,\_\}\cup\{a,b,c,d,e,f,g,h,i,j,k,l,m,n,o,p,q,r,s,t,u,v,w,x,y,z,ä,ö,ü,ß\} & \cup\\
\{A,B,C,D,E,F,G,H,I,J,K,L,M,N,O,P,Q,R,S,T,U,V,W,X,Y,Z,Ä,Ö,Ü\}. & \\
\end{split}
\end{equation}
In this document when we use the word alphabet -- if not otherwise stated -- we always think about the set $\Omega$. It is trivial from the construction of the sets above that $\Delta^+\subset\Delta\subset\Omega$ and $\Delta^+\subset\Delta\subset\mathbb{N}$ where $\mathbb{N}$ denotes the set of natural numbers (including 0).

\begin{remark}
The last statement is not 100\% correct here because elements of $\Delta$ are just symbols not actual numbers, but we allow us this slight incorrectness because our notation is chosen so that it is clear that f.i. the digit $9\in\Delta$ symbolizes the number $9\in\mathbb{N}$.
\end{remark}
 
\begin{mydef}
Elements of the set $$\mathcal{W}_n(\Omega)=\underbrace{\Omega\times \Omega\times\ldots \times \Omega}_{n}$$ are called \emph{words of length n over the alphabet $\Omega$}. We define the \emph{set of words over the alphabet $\Omega$} as $$\mathcal{W}(\Omega)=\bigcup_{n=1}^{\infty}\mathcal{W}_n(\Omega).$$
\end{mydef}

\begin{mydef}
Let $w\in\mathcal{W}({\Omega})$. It is easy to prove that there exists a unique $n>0$ such that $w\in\mathcal{W}_n({\Omega})$. $n$ is called the \emph{length} of the word $w$ and it is denoted by $|w|$.
\end{mydef}

\begin{mydef} 
Elements of the set
$$\mathcal{I}=\mathcal{W}(\Delta)$$
are called \emph{integer words} whereas elements of the set
$$\mathcal{I}^+=\mathcal{I}\setminus\mathcal{W}(\{0\})$$ are called \emph{positive integer words}.
\end{mydef}

Some examples of integer words are $(1,0)$, $(1,0,2)$ or $(0,0,1,2)$. It is obvious that integer words got their name from the fact that they symbolize integer numbers, for instance
\begin{equation}
\begin{split}
(1,0)\mapsto & 10^0\cdot 0+10^1\cdot 1=10\\
(1,0,2)\mapsto & 10^0\cdot 2 + 10^1\cdot 0 + 10^2\cdot 1=102\\
(0,0,1,2)\mapsto & 10^0\cdot 2 + 10^1\cdot 1 + 10^2\cdot 0 + 10^3\cdot 0=12
\end{split}
\end{equation}

\begin{mydef}
Let $w\in\mathcal{I}$ and let $n=|w|$. Then there exists $d_1,d_2,\ldots,d_n\in\Delta\subset\mathbb{N}$ such that $w=(d_1,d_2,\ldots,d_n)$. The number $$w^{\ast}=\sum_{i=0}^{n-1}10^id_{n-i}$$ is called the \emph{integer value} of $w$.
\end{mydef}

\begin{mydef}
Let $A,B\subset\mathcal{W}$ be arbitrary sets. We call the set 
$$A.B=A\times\{.\}\times B\subset\mathcal{W}$$
the \emph{dotproduct} of those sets.
\end{mydef}

In the upcoming subsections we define some special subsets of $\mathcal{W}$.

\subsection{$\mathcal{T}$ - the set of timestamps}
I will give an exact definition later, for now it is enough to know that $\mathcal{T}\subset\mathcal{W}$.
\subsection{$\mathcal{S}$ - the set of subjects}
\begin{mydef}
The \emph{set of subjects} is defined as
\begin{equation}
\mathcal{S}=\{(L,e,s,e,n),(S,c,h,r,e,i,b,e,n),(S,c,h,p,r,a,c,h,e),(M,a,t,h,e)\}.
\end{equation}
\end{mydef}

\subsection{$\mathcal{L}$ - the set of levels}
\begin{mydef}
The \emph{set of levels} is defined as
\begin{equation}
\mathcal{L}=\{(E,i,n,f,a,c,h),(M,i,t,t,e,l),(S,c,h,w,e,r)\}.
\end{equation}
\end{mydef}

\subsection{$\mathcal{K}$ - the set of tasks}
I need more details in order to give an exact definition, for now it is enough to know that $\mathcal{K}\subset\mathcal{W}$.

\subsection{$\mathcal{U}$ - the set of subtasks}
I need more details in order to give an exact definition, for now it is enough to know that $\mathcal{U}\subset\mathcal{W}$.

\subsection{$\mathcal{A}$ - the set of alpha ids}
\begin{mydef}
Consider the following recursion of sets in $\mathcal{W}:$
\begin{equation}
\begin{split}
\mathcal{A}_0= & \mathcal{I}^+\\
\vdots &\\
\mathcal{A}_{n} = &\mathcal{A}_{n-1}.\mathcal{I}^+ \mbox{ for } n>1.\\
\end{split}
\end{equation}
The set $$\mathcal{A}=\bigcup_{n=0}^{\infty}\mathcal{A}_n\subset\mathcal{W}$$ is called the \emph{set of alpha ids}. 
\end{mydef}

Some examples of alpha ids are $(1,.,2,.,10)$, $(10,.,2)$ or $(1,.,2,.,0,2)$, but f.i.  $(10,.,0,.,9,.,10)$ or $(a,.,10,.,b)$ are not valid alpha ids. 

\begin{mydef}
For each $w\in\matchal{A}$ there exists a unique $n>0$ such that $w\in\mathcal{A}_n$. $n$ is called the dimension of $w$ and it is denoted by $\dim(w)$.
\end{mydef}

\begin{mydef}
Let $w\in\mathcal{A}$ and let $n=\dim(w)$. Then there exists $I_1,I_2,\ldots,I_n\in\mathcal{I}^+$ such that $w=(I_1,.,I_2,.,\ldots,.,I_n)$. The vector 
$$w^{\times}=(I_1^{\ast},I_2^{\ast},\ldots,I_n^{\ast})\in\mathbb{N}^n$$ is called the \emph{integer decomposition} of $w$.
\end{mydef}

It can be shown that one can define a total order on $\mathcal{A}$ using the following order relation.

\begin{mydef}
Let $\alpha, \beta\in\mathcal{A}$. We say that $\alpha\prec\beta$ if and only if
\begin{itemize}
\item there exists $0<m<\min(\dim\alpha,\dim\beta)$ such that $\alpha^{\times}_i=\beta^{\times}_i$ for all $i<m$ but $\alpha^{\times}_m<\beta^{\times}_m$ or
\item $\dim\alpha < \dim\beta$ and $\alpha^{\times}_i=\beta^{\times}_i$ for all $i\leq\dim\alpha$.
\end{itemize}
\end{mydef}

\section{Mark tables}
Let $M=\{0,1\}$ and let us fix $\mathcal{F}\subset\mathcal{A}$, $|\mathcal{F}|<\infty$ as the set of \emph{feasible alpha ids}\footnote{This is an abstraction of the alpha ids in the \texttt{alphalist.xml} file}. We construct the set $\mu$ as 
$$\mu=\underbrace{\mathcal{T}}_{\mbox{timestamps}}\times\underbrace{\mathcal{S}}_{\mbox{subjects}}\times\underbrace{\mathcal{L}}_{\mbox{levels}}\times\underbrace{\mathcal{K}}_{\mbox{tasks}}\times\underbrace{\mathcal{U}}_{\mbox{subtasks}}\times\underbrace{\mathcal{F}}_{\mbox{alphaids}}\times \underbrace{B}_{\mbox{marks}}.$$ 
A \emph{mark table} is an element in $\mathcal{W}(\mathcal{\mu})$\footnote{I must emphasize earlier in the document that the notation $\mathcal{W}$ can be used not only for alphabets!}, but we can also think about it as a finite sequence $M=(m_i)_{i\in I}$ in $\mu$ with $I=\{1,2,\ldots,n\}$ for some $n<\infty$.

Before we go on we introduce some further notations related to finite, totally ordered sets.
\begin{mydef}
If $(R,\prec)$ is a finite, totally ordered set, then the sequence $R^{\prec}=(r^{\prec}_i)_{i=1}^{|R|}$ defined by the recursion
\begin{equation}
\begin{split}
r^{\prec}_1= & \min R\\
\vdots &\\
r^{\prec}_n = & \min R\setminus\{r^{\prec}_1,r^{\prec}_2,\ldots,r^{\prec}_{n-1}\} \mbox{ for } n>1.\\
\end{split}
\end{equation}
is called the \emph{increasing sequence} of $R$. Similarly, $R^{\succ}$ can be defined as the \emph{decreasing sequence}.
\end{mydef}

\section{Evaluation modes}
We present two different evaluation methods here, one is called ``Das kann ich!'' (\emph{mode A1}) and the other is called ``Das kann ich bald wenn ich moch ein wenig übe'' (\emph{mode A2}). The evaluation methods are basically mappings from $\mathcal{W}({\mathcal{M}})$ to the set of finite sequences over $\mathcal{F}$.

\subsection{Description of evaluation mode A1}
We start from a mark table $M=(m_i)_{i\in I}$. Let's define the relation $\sim$ over $I$ as follows:
\begin{equation}
i\sim j \Leftrightarrow m_i^{(6)}=m_j^{(6)}.
\label{eq:part}
\end{equation}
It is easy to see that (\ref{eq:part}) is an equivalence relation and as such it defines a partition over $I$, i.e. there exist some nonempty, disjoint sets $I_1,I_2,\ldots,I_r$ such that $$I=\cup_{i=1}^r I_i$$ and $i$ and $j$ are in $I_k$ if and only if $i\sim j$ i.e. $m_i^{(6)}=m_j^{(6)}$. Moreover, this resolution of $I$ is unique. Let us choose an element $i_k\in I_k$ (i.e. a \emph{representative}) for each $k=1,2,\ldots,r$.

Let $t\in [0,100]$ denote our predefined threshold for fullfilling a competency. Then, our preliminary result set is defined by 
$$R=\{m_{i_k}^{(6)}: |I_k|^{-1}\sum_{i \in I_k}m_{i}^{(7)}\geq 100^{-1}t, (k=1,2,\ldots r)\}.$$
Our final result is then $R^{\prec}$.
\begin{remark}
In our implementation we do not allow our result sequence to be longer then a predefined number $L\in\mathbb{N}^+$, so we take the sequence $(r^{\prec}_i)_{i=1}^{\min(|R|,L)}$ instead.    
\end{remark}

\begin{remark}
It is possible, that $R=\emptyset$, especially when the thresold $t$ is too high. In that case $R^{\prec}$ also becomes empty of course.
\end{remark}

\begin{remark}
Note that we could have projected or sequence into $\mathcal{F}\times\mathcal{B}$ because the other coordinates where not used at all in the evaluation.
\end{remark}
\subsection{Description of evaluation mode A2}
Assume that we have the marking table $M=(m_i)_{i\in I}$ in hand which is to be evaluated. At first We construct the set $I_0=\{i\in I: m_i^{(7)}=0\}\subset I$. Using the equivalence relation of \ref{eq:part} we can find nonempty, disjoint sets such that $$I_0=\cup_{i=1}^rI_i.$$ Let us choose an element $i_k\in I_k$ (i.e. a \emph{representative}) for each $k=1,2,\ldots,r$. Then, our preliminary result set is defined by
$$R=\{m^{(6)}_{i_k}:k=1,2,\ldots,r\}.$$
Our final result set is then $R^{\prec}$.
\begin{remark}
In our implementation we do not allow our result sequence to be longer then a predefined number $L\in\mathbb{N}^+$, so we take the sequence $(r^{\prec}_i)_{i=1}^{\min(|R|,L)}$ instead.    
\end{remark}
\begin{remark}
Note that we could have projected or sequence into $\mathcal{F}\times\mathcal{B}$ because the other coordinates where not used at all in the evaluation.
\end{remark}
\begin{remark}
If the mark received is 1 for each subtask, i.e. $m^{(7)}_i=1$ for all $i$ then $I_0=\emptyset$. In that case $R$ and consequently $R^{\prec}$ becomes empty too.
\end{remark} 

\end{document}
