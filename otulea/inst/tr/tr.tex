\documentclass{article}
\usepackage[a4paper,landscape,left=5mm,right=5mm,top=10mm,bottom=10mm]{geometry}
\usepackage{multirow}
\usepackage[utf8]{inputenc}
\usepackage[table]{xcolor}
\usepackage{hhline}
\usepackage{rotating}
\usepackage{tabularx}
\usepackage{array}
%\usepackage{booktabs}
%\usepackage[thinlines]{easytable}
\usepackage{tikz}
\usetikzlibrary{shapes,decorations}
%\usepackage{amsmath,amssymb}
\usepackage{dingbat}
%\usepackage{framed}
\usepackage{verbatim}
%\usepackage{array,booktabs}
% color definitions
\definecolor{Lesen-10}{RGB}{199,237,226}
\definecolor{Lesen-5}{RGB}{134,199,190}
\definecolor{Lesen-0}{RGB}{96,148,125}
\definecolor{Schreiben-10}{RGB}{209,233,202}
\definecolor{Schreiben-5}{RGB}{154,204,148}
\definecolor{Schreiben-0}{RGB}{106,166,106}
\definecolor{Sprache-10}{RGB}{246,246,213}
\definecolor{Sprache-5}{RGB}{234,234,119}
\definecolor{Sprache-0}{RGB}{223,196,11}
\definecolor{Mathe-10}{RGB}{248,230,174}
\definecolor{Mathe-5}{RGB}{230,188,125}
\definecolor{Mathe-0}{RGB}{180,136,86}
\definecolor{frame}{RGB}{185,186,150}

\newcommand{\specialcell}[1]{%
  \begin{tabular}{@{}c@{}}#1\end{tabular}}
\newcommand{\doboz}[2]{
\begin{tikzpicture}
  \node [mybox] (box){%
    \parbox{.2\textwidth}{
      \vspace{5pt}
      #2
    }
};
\node[fancytitle, right=10pt] at (box.north west) {#1};
\end{tikzpicture}
}

%\newcommand{\specialcell}[1]{%
%  \vtop{\vskip0pt\hbox{\begin{tabular}{@{}c@{}}#1\end{tabular}}%}}

%\newcommand*{\tabbox}[2][t]{%
%    \vspace{0pt}\parbox[#1][3.7\baselineskip]{1cm}{\strut#2\str%ut}}

\usepackage{Sweave}
\begin{document}
\thispagestyle{empty}

% Define box and box title style
\tikzstyle{mybox} = [draw=black,rectangle, rounded corners, inner sep=5pt, inner ysep=5pt]
\tikzstyle{fancytitle} =[fill=white, text=black,draw=black]


%\begin{tabular}{|l|l|l|}
%\multicolumn{1}{l}{} & \multicolumn{1}{l}{Teilnehmer/Teilnehmerin: \textbf{X0AT2}} & \multicolumn{1}{r}{Datum: 14. 2. 2012.}\\
%\hhline{~|--}
%\cline{2-3}
%\multicolumn{1}{l|}{} &\cellcolor{frame}Lesen & \cellcolor{frame}{Schreiben}\\
%\hline
%\cellcolor{frame}\vtop{\vskip0pt\hbox{\begin{sideways}\parbox{9em}{\centering Kannbeschreibungen erfüllt}\end{sideways}}}
%&

\begin{tabular}{|m{6mm}|l|l|l|l|}
\multicolumn{1}{l}{} & \multicolumn{3}{l}{Teilnehmer/Teilnehmerin: \textbf{X0AT2}} & \multicolumn{1}{r}{Datum: 14. 2. 2012.}\\
\hhline{~|----}
\multicolumn{1}{l|}{} &\cellcolor{frame}Lesen & \cellcolor{frame}{Schreiben} & \cellcolor{frame}Sprache & \cellcolor{frame}Mathe\\
\hline
\cellcolor{frame}
\vspace{.5em}
\begin{sideways}\parbox{9em}{\centering Kannbeschreibungen teilweise erfüllt}\end{sideways} & 
\cellcolor{Lesen-5}
\specialcell{      
  \doboz{1.2.2 (1)}{\leftpointright Kann Wörter mit ansteigender Komplexität (Konsonantenhäufung) recodieren und decodieren.}\\
  \doboz{1.3.2 (1)}{\leftpointright Kann Wörter mit ansteigender Komplexität (Konsonantenhäufung) recodieren und decodieren.}\\
  \doboz{1.3.3 (1)}{\leftpointright Kann Sätze mit ansteigender Länge sinnerfassend lesen.\\\leftpointright Kann SPO-Sätze und SPO-0 Sätze mit Einfügungen sinnerfassend lesen.}
}
& 
\cellcolor{Schreiben-5}
\specialcell{
\doboz{2.1.1 (2) $\Uparrow$}{\leftpointright Kann lautierte einzelne Laute verschriftlichen.}\\
\doboz{2.1.3 (2) $\Rightarrow$}{\leftpointright Kann Groß- und Kleinbuchstaben in Druckschrift unterscheiden.}\\
\doboz{2.1.4 (2) $\Uparrow$} {\leftpointright Kann Wörter mit Silben, die aus einem Vokal oder Diphtong bestehen schreiben (O-ma, Au-to).}
}
&
\cellcolor{Sprache-5}
\specialcell{      
  \doboz{1.2.2 (1)}{\leftpointright Kann Wörter mit ansteigender Komplexität (Konsonantenhäufung) recodieren und decodieren.}\\
  \doboz{1.3.2 (1)}{\leftpointright Kann Wörter mit ansteigender Komplexität (Konsonantenhäufung) recodieren und decodieren.}\\
  \doboz{1.3.3 (1)}{\leftpointright Kann Sätze mit ansteigender Länge sinnerfassend lesen.\\\leftpointright Kann SPO-Sätze und SPO-0 Sätze mit Einfügungen sinnerfassend lesen.}
}
&
\cellcolor{Mathe-5}
\specialcell{
\doboz{2.1.1 (2) $\Uparrow$}{\leftpointright Kann lautierte einzelne Laute verschriftlichen.}\\
\doboz{2.1.3 (2) $\Rightarrow$}{\leftpointright Kann Groß- und Kleinbuchstaben in Druckschrift unterscheiden.}\\
\doboz{2.1.4 (2) $\Uparrow$} {\leftpointright Kann Wörter mit Silben, die aus einem Vokal oder Diphtong bestehen schreiben (O-ma, Au-to).}
}
\\
\hline
\end{tabular}


\end{document}
